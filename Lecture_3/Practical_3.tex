\input{Header.tex}
\title{MATLAB Lecture 3 - Practical}
\date{November 2013}
\begin{document}

\maketitle

\section{Covariance matrix and QR decomposition}

Write the covariance matrix C in terms of Q and/or R.

\section{Implement polyfit}

In this practical, you will write your own implementation of MATLAB's
\code{polyfit} function. Start by reviewing MATLAB's help for \code{polyfit}
to remind yourself of how it works and what values it returns.

\begin{verbatim}
>> help polyfit
\end{verbatim}


We will start with a simple implementation and add more features later.
The simplest version of \code{polyfit} would look like this:

\begin{verbatim}
function [ P ] = mypolyfit(x,y,n)

% Assume that x, y are already column vectors.
% n == Degree of polynomial to fit.

X = ones(size(x));

for i = 1:n
    X = [  X  x.^i  ];
end

P = X \ y;
\end{verbatim}

The next step is to test this program against MATLAB's version and see if
it produces similar fits. Think about how you might test the function.
You will have to generate a few sample (x,y) data sets with known functions
and see how well the fit works. For example, you might consider using the
following:

\begin{verbatim}
x = [1:10]';

y1 = 3*x;
y2 = x.^2;
y3 = 2 + 3*x + 4*x.^2;
\end{verbatim}

Then compare the fits between your function and MATLAB's:

\begin{verbatim}
P =   polyfit(x, y1, 1) % MATLAB polyfit;  1 == polynomial of degree 1.
P = mypolyfit(x, y1, 1) % Your polyfit;    1 == polynomial of degree 1.

P =   polyfit(x, y2, 2) % MATLAB polyfit;  2 == polynomial of degree 2.
P = mypolyfit(x, y2, 2) % Your polyfit;    2 == polynomial of degree 2.

P =   polyfit(x, y3, 2) % MATLAB polyfit;  2 == polynomial of degree 2.
P = mypolyfit(x, y3, 2) % Your polyfit;    2 == polynomial of degree 2.
\end{verbatim}

Chances are that there will be some differences between the results from
\code{mypolyfit} and MATLAB's \code{polyfit}. Think about the results and
consider whether the differences seem reasonable.

Now use the \code{rand} or \code{randn} functions to add some noise
to the dataset. How would you do that? Try the fits again and check
that MATLAB's \code{polyfit} and your own \code{mypolyfit} give
reasonable estimates for the function coefficients.

Do \code{mypolyfit} and \code{polyfit} give more similar results after
you add random errors? Why does that happen?

\subsection{Covariance matrix}

This is the formula to calculate the covariance matrix:

\[
	C = (X^T X)^{-1} \sigma^2
\]

Use the QR decomposition as discussed in class to write this function
in terms of $R$ only. The \code{qr} function from MATLAB returns the
QR decomposition of a matrix.

\begin{verbatim}
[ Q R ] = qr(M)
\end{verbatim}

Use this function, along with the lecture notes, to modify \code{mypolyfit}
so that it also computes the covariance matrix $C$. To do this, you will
need to also compute $\sigma^2$ as described in the lecture notes.

\begin{verbatim}
function [ P C  ] = mypolyfit(x,y,n)

...
\end{verbatim}

What happens when $X$ is a square matrix? Why is that a problem for computing
the covariance matrix? The way that MATLAB deals with this problem is by not
returning the covariance matrix at all. Instead, it returns a MATLAB object
$S$ which can be used to produce the covariance matrix. Review the help on
\code{polyfit} to see what this object contains. Can you modify your
\code{mypolyfit} function to behave more like MATLAB's?

\begin{verbatim}
function [ P S  ] = mypolyfit(x,y,n)

...
\end{verbatim}

\begin{itemize}
\item Hint 1: \code{normr} is the norm of the residuals:
$\abs{\vec{\epsilon}}^2 = \sum \epsilon_i^2 = \vec{\epsilon} \cdot \vec{\epsilon}$ \\
(where $\cdot$ is the dot product).\\
You can use \code{e = y - X*P} and the MATLAB function \code{dot}.
\item Hint 2: \code{df} is the ``degrees of freedom'', (number of rows minus columns).
\end{itemize}


Once the covariance matrix has been computed, you can obtain the errors
on each of the model parameters. Modify \code{mypolyfit} so that it
also returns on the parameters if possible.

\begin{verbatim}
function [ P S err ] = mypolyfit(x,y,n)

...
\end{verbatim}




\section{Coma cluster}

Say something about the Coma galaxy cluster and maybe show a picture.
Explain the format of the data file.


\begin{verbatim}

data = dlmread('coma.txt');

ra  = data(:,1); % Right Ascention.
dec = data(:,2); % Declination.
x   = data(:,3); % X Position.
y   = data(:,4); % Y Position.
mag = data(:,5); % Magnitude.
vr  = data(:,6); % Radial velocity.

%
% Plot the positions of the galaxies.
%
plot(x,y,"r.")

\end{verbatim}











\end{document}





